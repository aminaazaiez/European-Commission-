
\documentclass[ 11pt]{article}
\usepackage[utf8]{inputenc}
\usepackage[style = authoryear, url=false]{biblatex}
\addbibresource{biblio.bib}
\begin{document}


Policy makers have a number of motives to foster participation of stakeholders in the policy process. They provide expert knowledge, information on encompassing interest but also legitimacy to the policy institution \parencite[see in particular][]{bouwen2002corporate,kohler2007institutional,de2016pressure}. The latter is particularly important for supra-national institutions like the European Union, which can be considered to carry a democratic deficit \parencite{follesdal2006there}. As a matter of fact, according to \parencite{coen2009lobbying}, European institutions have been "active in the creation of  a number of societal and environmental interest groups" \parencite[see also][]{kohler2007institutional, kluver2013lobbying}. Yet, the relation between stakeholders and policy makers is not unidirectional. As emphasized by \parencite{bouwen2002corporate}, it should rather be seen as an exchange of access goods, such as information and legitimacy, against influence on the policy process.

The interactions between interest groups and policy makers thus gives rise to a complex lobbying network \parencite{coen2009lobbying} in which influence can flow through different channels, e.g. in the case of European politics through national governments or directly at the European level \parencite{pappi1999organization}, on different institutions, e.g. executive or legislative, and in different phases of the policy process: agenda-setting, policy formulation or implementation  \parencite[see e.g.][]{dur2007inclusion}. In this complex framework, a perceived danger is that certain actors exert a disproportionate impact on the policy process. Measuring influence has thus been a key issue in the political science focusing on lobbying \parencite[see e.g.][]{mahoney2007lobbying,dur2008measuring}. However,  progress in this field has long been plagued by lack of data and transparency of the lobbying processes. In the European context, the introduction of the transparency register \parencite{greenwood2013transparency} and the raising role of public consultations \parencite{schmidt2013democracy} in which the opinion of  interest groups is expressed publicly and archived electronically are potential game-changers. A burgeoning literature \parencite[e.g.][]{kluver2009measuring,kluver2011contextual,kluver2013lobbying,kluver2015promises,beyers2014policy} has applied machine learning techniques to classify documents from public consultations according to the opinions expressed and measure their influence through their distance to the output of the policy process. Two caveats that this literature must face is the difficulty to train classifiers given the relatively small size of the datasets \parencite[see e.g.][]{bunea2015quantitative} and the difficulty to express the opinion in a document through a uni-dimensional value. This paper approach these issues by proposing a complementary method to analyze open consultation data: the construction of multi-dimensional policy network maps through crowdsourcing. 
\printbibliography
%\bibliographystyle{unsrt}
%\bibliography{biblio}
\end{document}

